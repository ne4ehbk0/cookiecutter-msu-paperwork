%!TEX root = {{cookiecutter.project_name}}.tex


% Добавьте ссылку на файлы с текстом работы
% Можно использовать команды:
%   \input или \include
% Пример:
%    \input{mainfiles/1-section} или \include{mainfiles/2-section}
% Команда \input позволяет включить текст файла без дополнительной обработки
% Команда \include при включении файла добавляет до него и после него команду
% перехода на новую страницу. Кроме того, она позволяет компилировать каждый файл
% в отдельности, что ускоряет сборку проекта.
% ВАЖНО: команда \include не поддерживает включение файлов, в которых уже содержится команда \include,
% т.е. не возможен рекурсивный вызов \include
\newcommand*{\Source}{
    %!TEX root = ../{{cookiecutter.project_name}}.tex
\phantomsection
\section*{Постановка задачи}
Цель работы состоит в изучении наиболее простой антенны — линейного симметричного вибратора, а также ознакомлении с её несимметричной версией — штыревой
антенной. В ходе выполнения лабораторной работы Вы ознакомитесь с возможностями
программы моделирования проволочных антенн MMANA..
\addcontentsline{toc}{section}{Постановка задачи}
    %%!TEX root = ../{{cookiecutter.project_name}}.tex

\section{Криптосистема Мак-Элиса}

    %%!TEX root = ../{{cookiecutter.project_name}}.tex

\section{Ключевое пространство криптосистемы Мак-Элиса.}

    %\include{mainfiles/4-mceliece-sidelnikov}
    %%!TEX root = ../{{cookiecutter.project_name}}.tex

\section[Множество открытых ключей криптосистемы Мак-Элиса---Си\-дель\-ни\-кова]{Множество открытых ключей криптосистемы Мак-Элиса---Сидельникова}


    %%!TEX root = ../{{cookiecutter.project_name}}.tex

\section{Классы эквивалентных ключей в случае \texorpdfstring{$u=2$}{u=2}.}


}


% Информация о годе выполнения работы
\def\Year{%
    % 2006%
    \the\year%     % Текущий год
}

% Укажите тип работы
% Например:
%     Выпускная квалификационная работа,
%     Магистерская диссертация,
%     Курсовая работа, реферат и т.п.
\def\WorkType{%
    % Выпускная квалификационная работа%
    % Магистерская диссертация%
    % Курсовая работа%
    % Реферат%
    Отчёт%
}
\def\ReportNum{%
    по лабораторной работе №1%
}
% Название работы
%%%%%%%%%%% ВНИМАНИЕ! %%%%%%%%%%%%%%%%
% В МГУ ОНО ДОЛЖНО В ТОЧНОСТИ
% СООТВЕТСТВОВАТЬ ВЫПИСКЕ ИЗ ПРИКАЗА
% УТОЧНИТЕ НАЗВАНИЕ В УЧЕБНОЙ ЧАСТИ
\def\Title{%
    Дипольные (вибраторные) антенны%
}


% Имя автора работы
\def\Author{%
    Гасюк Алексей Андреевич%
}
\def\AuthorShort{%
    А.~А.~Гасюк%
}
\def\StudentGroupNumber{%
	Студент гр. 591%
}

% Информация о научном руководителе
%% Фамилия Имя Отчество%
\def\SciAdvisor{%
    Щербинин Всеволод Владиславович%
}
%% В формате: И.~О.~Фамилия%
\def\SciAdvisorShort{%
    В.~В.~Щербинин%
}
%% должность научного руководителя
\def\Position{%
    % профессор%
    доцент%
    % старший преподаватель%
    % преподаватель%
    % ассистент%
    % ведущий научный сотрудник%
    % старший научный сотрудник%
    % научный сотрудник%
    % младший научный сотрудник%
}
%% учёная степень научного руководителя
\def\AcademicDegree{%
    % д.ф.-м.н.%
    % д.т.н.%
    к.ф.-м.н.%
    % к.т.н.%
    % без степени%
}

% Информация об организации, в которой выполнена работа
%% Город
\def\Place{%
    Барнаул%
}
\def\Ministry{%
     Министерство науки и высшего образования Российской Федерации%
}
%% Университет
\def\Univer{%
    Алтайский государственный университет%
}
%% Факультет
\def\Faculty{%
    Институт цифровых технологий, электроники и физики%
}
%% Кафедра    
\def\Department{%
    Кафедра радиофизики и теоретической физики%
}     

%%%% Переключите статус документа для отладки
%%%% В режиме draft документ собирается очень быстро
%%%% и выводится полезная информация о том
%%%% какие строки вылезают за границы документа, что удобно для борьбы с ними
\def\Status{%
    % draft%
    final%
}

%%%% Включает и выключает подпись <<С текстом работы ознакомлен>>
\def\EnableSign{%
    % true%
}

